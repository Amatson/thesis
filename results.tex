%!TEX root = thesis.tex
%% %% ***************** Results *****************


\section{Results}\label{sec:results}

This section presents the results of HML pipeline combinations.
The values of algorithm evaluations are compared
and the results are judged.
Before that,
we discuss about the memory issue encountered
that affected our pipeline component choices.


%% ************************************************************************************************************


\subsection{Memory issues and limitations}\label{subsec:res-memory-issues}
Memory is crucial resource in ML training.
Algorithms take multiple steps while iterating the data
and intermediate results are stored in the RAM rather than on the disk.
While building ML pipeline in Azure ML Studio,
a memory issue emerged
that affected several components
and caused serious limitations
in terms of usable components and data size.
Due to the time limits of this study,
this issue was not resolved
and the problem causing it was not found.
As several conditions
considering the environment costs
were already issued by the company,
the issue was supposed to be linked with
compute instance property limitations
and thus could not be resolved without cost overruns.
However, this was not certain.


\todo{Limitations to components. List effects of this problem. This subsection is still work-in-progress}
%% TODO: limitations to components
%% limitations to data size
%% effects on choises in components?






During the initial pipeline runs
the execution came to an abrupt stop
and Azure notified about memory issues. %% TODO: Add reference to issue
These problems were linked to the data amount
which had to be reduced to 600 megabytes
before any pipeline could be finished using the data.
This reduction was against the initial goal
where preferably all the data could have been used.

Considerable amount of time was used
to fix or avoid this issue
but nothing clear was found
that would explain the error received.
While working with the issue
it was also noted
that data needed more cleaning
in order to ease the preprocessing phase
as described with more detail in section~\ref{subsec:meth-data-anonymization}
Thus,
the data had to be imported from log archive
and anonymized once more.

To advance the study more efficiently
it was decided to trim info-type log messages from data
hence reducing the data amount considerably.
Final data included 8.6 million log rows
which was about 10\% of the original data size.
Before final cleaning operations
the data took 8.1GB od disk space,
and after cleaning the rawmessage-field
the final disk size was 6.6GB\@.
Even with this data size,
some Azure ML components faced this memory issue
and forced us to choose such components
that were able to handle these data amounts.

%% TODO: Move some parts of the above memory problem elsewhere
%%

%% TODO: Results: Feature Hashing runs out of memory in Model Training with full data (unconventional training)
%% Trying with smaller splits, which, once again, reduces data amount.

Even though with rawmessage we were forced to reduce the data amount significantly,
it was seen as valuable data source
as it included more data than pure message had in JSON structure.
As data amount was skimmed to less than 1\% of the original amount,
the results may not be reliable.


%% ************************************************************************************************************

\subsection{Algorithm estimation results}\label{subsec:res-algorithm-estimation-results}

Based on the first pipeline experiments,
the Decision Forest Regression algorithm gave the most promising results,
so the values listed in this section are based on the use of this algorithm unless otherwise stated.
However, this first impression was a misinterpretation,
as the final results show.
For unknown reasons,
when time frame compression was carried out with SQL queries instead of R-script,
the results were considerably better.
However, these results could not be relied upon,
as further investigation showed that the time frame compression was incorrect.
This was verified by calculating the total number of log rows
and comparing it to the sum of the line counts in all time frames.
As something did not work as intended with SQL queries,
SQL-based time frame compression was decided to reject.
It was also much easier to include more statistical metrics
with R-scripting than with SQL queries.

As explained in the section~\ref{subsec:pipe-branching},
the amount of pipeline component combinations to compare was significant.
It was decided to run initial experiments with a few basic component combinations
and select the most promising branches to continue with.
For this reason,
it was decided to reject pure n-gram feature extraction in the final analysis
because the memory issue affected so much its reliability in terms of the amount of data.
Feature hashing, however,
could be included.
The amount of data was seen sufficient to consider the results reliable,
although, to make this possible,
a lot of information was lost due to the level of compression used.

In the table~\ref{tab:results-decision-forest-all},
the final results of the pipelines using Decision Forest Regression algorithm in phase 2
are listed for comparison.
When interpreting the comparison metrics,
it was decided to focus on the coefficient of determination (CoD-value),
mean absolute error (MAE-value),
and their difference with the corresponding metrics
in cases where the anomaly probability value was omitted from phase 2.
Like we explained in the section~\ref{subsec:pipe-branching},
we assumed that anomaly probability values calculated by the anomaly detection algorithm
provides valuable information for ticket count estimating.
Thus,
if these values are removed before the phase 2 algorithm training,
the estimation capabilities of the algorithm should weaken.
Hence,
we decided to experiment more with those pipeline component combinations
which resulted in noticeably better results
than their anomaly-probability-omitted counterparts.

\begin{table}[htb]
    \begin{tabularx}{\textwidth}{|Y|L{0.091\textwidth}|L{0.096\textwidth}|L{0.077\textwidth}|L{0.077\textwidth}|L{0.105\textwidth}|}
        \hline
        \textbf{Combination} &
        \textbf{MAE} &
        \textbf{RMSE} &
        \textbf{RSE} &
        \textbf{RAE} &
        \textbf{CoD} \\ \hline
        Msg PropT           				& 3.6875				& 4.2886					& 1.6966					& 1.3169					& -0.6966		\\
        Msg PropT NA	     				& 3.5125				& 4.2718					& 1.6834					& 1.2544					& -0.6834		\\
        \hline
        Msg UnconT             				& 3.5236				& 4.3578					& 1.3493					& 1.1313					& -0.3493		\\
        Msg UnconT NA         				& 4.0405				& 4.8958					& 1.7030					& 1.2972					& -0.7030		\\
        \hline
        PreP.Msg PropT     					& 3.25  				& 3.7720					& 1.3125					& 1.1607					& -0.3125		\\
        PreP.Msg PropT NA            		& 3.5125				& 4.2718					& 1.6834					& 1.2544					& -0.6834		\\
        \hline
        PreP.Msg UnconT       				& 3.7905				& 4.5707					& 1.4843					& 1.2169					& -0.4843		\\
        PreP.Msg UnconT NA           		& 4.0405				& 4.8958					& 1.7030					& 1.2972					& -0.7030		\\
        \hline
        \textbf{Msg FHash PropT }   		& 2.4875				& 2.8611					& 0.7551					& 0.8883					& \textbf{0.24484}		\\
        Msg FHash PropT NA  				& 2.85  				& 3.1922					& 0.9400					& 1.0178					& \textbf{0.05990}		\\
        \hline
        Msg FHash UnconT        			& 2.6071				& 3.2484					& 0.7234					& 0.8488					& 0.27657		\\
        Msg FHash UnconT NA        			& 2.6562				& 3.2593					& 0.7282					& 0.8648					& 0.27171		\\
        \hline
        PreP.Msg FHash PropT	   			& 3.05  				& 3.4058					& 1.0701					& 1.0892					& -0.0701		\\
        PreP.Msg FHash PropT NA  			& 2.6   	 			& 2.8858					& 0.7682					& 0.9285					& 0.23172		\\
        \hline
        PreP.Msg FHash UnconT	       		& 2.7901				& 3.4467					& 0.8144					& 0.908						& 0.18556  		\\
        PreP.Msg FHash UnconT NA      		& 2.8616				& 3.4125					& 0.7983					& 0.9316					& 0.20162 		\\
        \hline
        \textbf{RawMsg UnconT }        		& 2.9   				& 3.8200					& 0.7859					& 0.7928					& \textbf{0.21402}		\\
        RawMsg UnconT NA              		& 3.0733				& 3.9588					& 0.8441					& 0.8402					& \textbf{0.15588}		\\
        \hline
        \textbf{RawMsg PropT}       		& \textbf{1.3611}		& 1.8263					& 1.6678					& 1.0208					& -0.6678		\\
        RawMsg PropT NA           			& \textbf{2.5833}		& 2.8694					& 4.1168					& 1.9375					& -3.1168		\\
        \hline
        \textbf{PreP.RawMsg UnconT}    		& 2.9466				& 3.7159					& 0.7437					& 0.8056					& \textbf{0.25627}		\\
        PreP.RawMsg UnconT NA             	& 3.0733				& 3.9588					& 0.8441					& 0.8402					& \textbf{0.15588}		\\
        \hline
        \textbf{PreP.RawMsg PropT }  		& \textbf{0.5277}		& 0.6328					& 0.2002					& 0.3958					& \textbf{0.79976}		\\
        PreP.RawMsg PropT NA        		& \textbf{2.5833}		& 2.8694					& 4.1168					& 1.9375					& \textbf{-3.1168}		\\
        \hline
        RawMsg FHash PropT					& 3.7916				& 5.3078					& 2.6361					& 1.3787					& -1.6361		\\
        RawMsg FHash PropT NA        		& 4.0625				& 5.2026					& 2.5326					& 1.4772					& -1.5326		\\
        \hline
        RawMsg FHash UnconT    				& 2.875 				& 3.3994					& 1.0458					& 1.0267					& -0.0458		\\
        RawMsg FHash UnconT NA         		& 2.9   				& 3.8249					& 1.3240					& 1.0357					& -0.3240		\\
        \hline
        PreP.RawMsg FHash PropT				& 2.8194				& 2.8993					& 1.4834					& 1.4097					& -0.4834		\\
        PreP.RawMsg FHash PropT NA    		& 2.5694				& 3.0503					& 1.6419					& 1.2847					& -0.6419		\\
        \hline
        PreP.RawMsg FHash UnconT			& 2.6153				& 3.4303					& 0.8327					& 0.9324					& 0.16723		\\
        PreP.RawMsg FHash UnconT NA	    	& 2.5384				& 3.1784					& 0.7149					& 0.9050					& 0.28503		\\
        \hline
    \end{tabularx}
    \caption{Results of HML pipeline with Decision Forest regression algorithm in the phase 2.
        \textbf{FHash} means \textit{Feature Hashing},
        \textbf{PropT} indicates \textit{proper training},
        \textbf{UnconT} that \textit{unconventional training} is done in phase 1,
        \textbf{PreP.} means that \textit{text preprocessing} has been used, and
        \textbf{NA} means that \textit{anomaly values has been removed} for comparison (NoAnomalies).
        The most promising comparison metrics and their component combinations are bolded.
    }
    \label{tab:results-decision-forest-all}
\end{table}

As highlighted in the table\ref{tab:results-decision-forest-all},
the most promising results have been received by using:
\begin{itemize}
    \setlength\itemsep{0pt}
    \setlength{\parskip}{0pt}
    \item message data, feature hashing and proper training
    \item rawmessage data and unconventional training
    \item rawmessage data and proper training
    \item rawmessage data, preprocessed, proper training, and
    \item rawmessage data, preprocessed, unconventional training.
\end{itemize}

We have made another assumption,
that with different algorithms
the same component combinations would give the best results.
Next,
these chosen combinations are tested with different regression algorithm in the phase 2.

\begin{table}[htb]
    \begin{tabularx}{\textwidth}{|Y|L{0.08\textwidth}|L{0.10\textwidth}|L{0.08\textwidth}|L{0.08\textwidth}|L{0.1\textwidth}|}
        \hline
        \textbf{Algorithm} &
        \textbf{MAE} &
        \textbf{RMSE} &
        \textbf{RSE} &
        \textbf{RAE} &
        \textbf{CoD} \\ \hline
        Poisson PropT				& 3.1045				& 3.4232				& 1.0810				& 1.1087				& -0.0810  \\
        Poisson PropT NA 			& 3.0647			 	& 3.3359				& 1.0266			 	& 1.0945				& -0.0266  \\
        \hline
        Poisson UnconT				& 2.7530				& 3.4857			 	& 0.8329				& 0.8963			 	& 0.16704  \\
        Poisson UnconT NA 			& 2.7412				& 3.3489			 	& 0.7688				& 0.8924			 	& 0.23113  \\
        \hline
        NeuralNet PropT				& 2.9549				& 3.5045				& 1.1330				& 1.0553				& -0.1330  \\
        NeuralNet PropT NA			& 2.9368				& 3.4668				& 1.1087				& 1.0488				& -0.1087  \\
        \hline
        NeuralNet UnconT			& 3.629					& 4.6548			 	& 1.4854				& 1.1817			 	& -0.4854  \\
        NeuralNet UnconT NA 		& 3.7330				& 4.7398			 	& 1.5401				& 1.2154			 	& -0.5401  \\
        \hline
        Boosted PropT				& 2.2013				& 2.8158			 	& 0.7314				& 0.7861				& 0.26852  \\
        Boosted PropT NA			& 2.1983				& 2.7899				& 0.7180				& 0.7851				& 0.28193  \\
        \hline
        Boosted UnconT				& 3.0881				& 3.7033			 	& 0.9402				& 1.0054			 	& 0.05976  \\
        Boosted UnconT NA 			& 2.9448				& 3.4691			 	& 0.8250				& 0.9587			 	& 0.17494  \\
        \hline
        Linear PropT				& 2.9491				& 3.1988				& 0.9439				& 1.0532				& 0.05602  \\
        Linear PropT NA				& 2.8327			 	& 3.0956				& 0.8840			 	& 1.0117			 	& 0.11597  \\
        \hline
        Linear UnconT				& 2.8672				& 3.5982			 	& 0.8876				& 0.9335			 	& 0.11238  \\
        Linear UnconT NA 			& 2.6735				& 3.3376			 	& 0.7637				& 0.8704			 	& 0.23628  \\
        \hline
        \textbf{DecFor PropT}	 	& 2.4875			 	& 2.8611				& 0.7551				& 0.8883				& \textbf{0.24484}  \\
        DecFor PropT NA				& 2.85  		 		& 3.1922				& 0.9400				& 1.0178				& \textbf{0.05990}  \\
        \hline
        DecFor UnconT            	& 3.5236				& 4.3578				& 1.3493				& 1.1313				& -0.3493   \\
        DecFor UnconT NA         	& 4.0405				& 4.8958				& 1.7030				& 1.2972				& -0.7030	\\
        \hline
    \end{tabularx}
    \caption{Results of HML pipeline with different algorithms used in phase 2.
        \textbf{Poisson} means \textit{Poisson regression},
        \textbf{NeuralNet} indicates \textit{Neural Network regression},
        \textbf{Boosted} means \textit{Boosted Decision Tree regression},
        \textbf{Linear} means \textit{Linear regression}, and
        \textbf{DecFor} means \textit{Decision Forest regression}.
        Feature hashing has been used with message-column in each case.
        Each algorithm is tested with unconventional (\textbf{UnconT}) vs. proper training (\textbf{PropT}),
        and with or without anomaly probability values from phase 1 (\textbf{NA} means NoAnomalies).
        The most promising results are bolded.
    }
    \label{tab:results-msg-fhash-algorithm-comparison}
\end{table}

In the table~\ref{tab:results-msg-fhash-algorithm-comparison},
we can see the results of different regression algorithms
with message data and feature hashing.
However,
as the results indicate,
comparison metrics do not improve with different algorithms.
Thus,
Decision Forest Regression is selected for the final validation
with fresh test data.

\todo{Couple of more tables coming with final results with fresh test data set!}


\todo{In final results, explain that mostly default component parameters were used, and show what they were}


%% ************************************************************************************************************



\clearpage