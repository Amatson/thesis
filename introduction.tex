%!TEX root = thesis.tex
%% %% ***************** Introduction *****************

\section{Introduction}\label{sec:introduction}

%% Leave page number of the first page empty
\thispagestyle{empty}
%% TODO: Remember references
Artificial intelligence (AI) and machine learning (ML)
has found their way to
more and more fields of business.
In banking business they are already used in
fraud detection, risk management and service recommendations.\cite{donepudi2017machine}
Even though these
modern big data utilizing technologies
are widely used abroad,
in Finnish banking field AI and ML are not popularly utilized.
Instead,
many self-acting solutions are being used
to streamline manual labor
which could be called intelligent,
but are merely highly automated processes
and thus cannot be included in the AI category.
One of these technologies used in Finnish banking systems
is Robotic Process Automation (RPA).

RPA operates \enquote{on the user interface of other computer systems
in the way a human would do},\cite{van2018robotic}
but is bounded by predefined operations
thus being prone to unforeseen situations
such as faulty input.
RPA, like generally all other software,
produces log to \enquote{register
the automatically produced and time-stamped documentation
of events, behaviors and conditions
relevant to a particular system}\cite{delarosa2018log}.


Oy Samlink Ab (Samlink from now on)
was founded in 1994
and is now owned by Kyndryl.
From the early years
and going by the name of Samcom
the company was owned by several Finnish banks
developing all sorts IT solutions to them.
Most notable product is Codeapp
that makes mobile authentication possible
for client bank end customers.

Besides banking,
Samlink develops multiple other IT solutions
to wide range of customers,
for example,
entertainment platform solutions for DNA\@.
Even though Samlink can be considered
a modern technology company
the most modern technologies has not yet been adapted
in tool variety used in development.
However,
robotic process automation has been adopted
and is actively used in some
banking solutions to reduce
the amount of manual labor required
from banking clerks.

Like all other IT companies,
Samlink has a lot of data that is produced
and moved through its systems daily.
And as a company offering continuous development
as well as product maintenance services,
Samlink also serves as a technical help desk
regarding the software solutions produced.
As no IT solutions comes without bugs or misbehaviour,
Samlink service desk has to use considerable amount of time
to resolve whether current tech help request
is due to the problems in programming.
In this process, machine learning is able to help.
%% TODO: This would need some clarification and sourcing.

<RPA in use, but not intelligent processes>

<ML and Azure>

\subsection{Background and motivation}\label{subsec:intro-background-and-motivation}
In the field of information technology
logging is one of the most important methods
in problem-solving,
be it software or operating system related.\cite{delarosa2018log}

Typically,
at least in Samlink processes,
logging is a bit more verbose
than it needs to be.
This is usually because when the problem occurs
it is easier to already have the verbose logs available
than trying to replicate the issue
after setting logging to more verbose mode.
Too verbose logging, however,
leads into two problematic issues.
First of, the size of log is huge
and finding the critical leads for
solving the problem in hand
takes more time.
Of course,
with more strict logging
pinpointing the issue from within the logs
would be faster,
but then again,
solving the problem with only critical error messages
could be more time-consuming
if crucial context is missing.
Secondly,
as software tends to see some issues it encounters
worth of several dozens of log rows,
some rows are not critical information
if the software is able to continue,
especially if it is able to retry the process after failure
and get the job done at the second time.
These issues make log parsing considerably laborious.


%% TODO: continue
\begin{verbatim}
<importance of logs>

<current state of samlink AI processes>

<Log format>

<Intelligent automation>
\end{verbatim}

\subsection{Research objectives}\label{subsec:intro-research-objectives}
This research aims to kickstart the machine learning application usage in Samlink operations.
Multiple obstacles need to be tackled
as most of the phases in this study has not yet been encountered inside the company.

First and foremost,
it is crucial to construct some basic form for log data
so it is usable by machine learning algorithms.
Log data forming is on of the key elements in automatic log parsing applications
as it is not for just machines
but also for people to read.
%% TODO: more

As today more and more concern is set on anonymization
not only due to GDPR,
the data used for machine learning must be sanitized.
Because of this,
one major objective is to create a clean dataset
that is safe to use in cloud environment
without creating concern around security and privacy issues.
In addition to this,
data must also be clean enough
so that machine learning algorithms
are able to process it.

Finally,
the main objective this research is aiming to answer
is whether it is possible to use machine learning algorithms
in such ways that combine support ticket timestamps
and software run log
so that it can predict
if new support ticket might be coming.
%% TODO: go on?

%% TODO: Parse these to somewhere nice!
Few things to consider:
\begin{verbatim}
  * Counting anomaly "probability" for individual log rows
  * Random delay! Solving issue by grouping
  * Hybrid ML with anomaly detection and regression
  => Can this give any results for forecasting?
  ?? Assumption to lean on:
    !! Ticket causing log rows are ANOMALIES in log data.
    TODO: explain why this assumption is necessary!!!
    - Log amount is not necessarily proportional to ticket creating error amount, even with info-types filtered out
    - Considerable amount of rows may be related to one, or several different, tickets
\end{verbatim}
%% TODO: it's not really probability. What is it?
%% TODO: Is it really regression? Is that what we should be using?

What really happens in this research:
\begin{verbatim}
<Anonymization scope>
<Azure ML studio setup and prerequisites>
<Log data and timestamp combining>
<Connection for data, ML estimates>
\end{verbatim}

\subsection{Scope}\label{subsec:intro-scope}

In order to limit the study
to feasible length and content
it is necessary to define the scope for the thesis.
As major part of the time consumed for the study
was used for data acquiring and anonymizing
yet the most value comes from
content considering machine learning,
the length of the content per section
does not reflect
the amount of time used for each section.

\subsubsection*{Data anonymization}
Anonymization in the context of this thesis
refers to data sanitization process
purposed to edit the data into
more secure form in the privacy point of view.
In this study we aim to
create a dataset usable in ML training.
In this respect
anonymization is not in the main focus of the study
but only treated as a sub-phase
of the data preprocessing in whole.
Nevertheless,
anonymization is from the privacy point of view
the most important phase of data preprocessing.

Keeping this in mind,
anonymization is covered rather superficially,
only enough to explain the reasons
behind actions taken during anonymization process.

\subsubsection*{Azure setup}
The ML training and result scoring
is done in Azure ML environment.
As part of the study aims to create
an initial guideline for
ML process commissioning
the Azure setup phase is documented in such detail
reflecting the importance of this information
for future developers
starting Azure ML projects.

%% TODO: what more?

\subsubsection*{Data requirements}
Data purity in a sense of
how easy it is to be used by ML algorithms
creates challenges at the beginning of ML training.
If data is not consistent,
has lots of missing values
or is formed in unanticipated way,
it requires considerable amount of preprocessing
slowing the training process
and causing errors in pipeline runs.

In order to create a baseline for Samlink ML projects
this study aims to give basic criterion
what is required from the data,
so it is easily parsable by ML algorithm.
%% TODO: better words for parsable?


\subsubsection*{Machine learning methods}
Several different machine learning algorithms exist
%% TODO: Source?
that are aimed for different applications in mind.
For example,
to make an algorithm that can predict
the price of an apartment listed\cite{winky}
we could be using linear regression
and in order to detect
possible cyber threats from network traffic\cite{ghanem}
a two-class support vector machine could be utilized.
These two methods are very different in usage
and has their pros and cons in different applications.

As different methods can be used in creative ways
in very different applications
depending on how the data is presented
and how the ML problem is formed,
this study focuses on
just a few easily approachable training methods.

When it comes to anomaly detection,
only principal component analysis (PCA)
is considered as
PCA-based Anomaly Detection component is usable
from existing anomaly detection algorithms
in Azure ML Studio.
As purely Azure ML Studio is used during this study,
no other anomaly detection algorithms are debated.

\subsection{Structure}\label{subsec:intro-structure}
At the beginning of this thesis
we opened the key concepts discussed in the study.
We also defined the
%% TODO: nope. talk about next sections. move stuff from earlier sections


%% Opinnäytteessä jokainen osa alkaa uudelta sivulta, joten \clearpage
%%
%% In a thesis, every section starts a new page, hence \clearpage
\clearpage
